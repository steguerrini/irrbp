\hypertarget{index_intro}{}\section{Introduction}\label{index_intro}
Welcome to IrrBP -\/ an Irrlicht-\/Bullet Physics wrapper.

I wrote those few class to help you intagrate physics into your game, without knowing Bullet Physics Engine. The integration is very simple, and the function are very (irrlicht)user-\/friendly. You don't need to worry about optimization and memory leaks...IrrBP is alreading doing this!

IrrBP comes out without any kind of memory leak, and its fully performant.

If you have any questions or suggestions, email me at \href{mailto:stefanoguerrini93@gmail.com}{\tt stefanoguerrini93@gmail.com}\hypertarget{index_IrrBPexample}{}\section{IrrBP Integration Example}\label{index_IrrBPexample}
This example, shows you how to simply integrate bullet physics into your code. Here is it:


\begin{DoxyCode}
#include <irrlicht.h>
#include <IrrBullet.h>

using namespace irr;
using namespace core;
using namespace video;
using namespace scene;


static CIrrBPManager * bulletmgr;
static ISceneManager* smgr;
class Receiver : public IEventReceiver
{
public:
                Receiver();
                virtual bool OnEvent(const SEvent& event);
private:
};
Receiver::Receiver()
{

}

bool Receiver::OnEvent(const irr::SEvent &event)
{

        if(event.EventType == EET_MOUSE_INPUT_EVENT)
        {
                if(event.MouseInput.Event == EMIE_LMOUSE_PRESSED_DOWN)
                {
                        ISceneNode * node = smgr->addCubeSceneNode(10,0,-1,smgr->
      getActiveCamera()->getPosition());
                        CIrrBPBoxBody * body = bulletmgr->addRigidBox(node,40);
                        irr::core::vector3df rot = smgr->getActiveCamera()->getRo
      tation();
                        irr::core::matrix4 mat;
        
                        mat.setRotationDegrees(rot);
                        irr::core::vector3df forwardDir(irr::core::vector3df(mat[
      8],mat[9],mat[10]) *120);

                        body->getBodyPtr()->setLinearVelocity(irrVectorToBulletVe
      ctor(forwardDir) * 2);
        
                }
                
        }

                                
        return false;
}
int main()
{

        

        IrrlichtDevice *device =
                createDevice(video::EDT_OPENGL, core::dimension2d<u32>(640, 480))
      ;
        Receiver * recv = new Receiver();
        if (device == 0)
                return 1; // could not create selected driver.
        
        video::IVideoDriver* driver = device->getVideoDriver();
         smgr = device->getSceneManager();

        device->getFileSystem()->addZipFileArchive("map-20kdm2.pk3");

        device->setEventReceiver(recv);

        scene::IAnimatedMesh* mesh = smgr->getMesh("20kdm2.bsp");
        scene::IMeshSceneNode* node = 0;

        if (mesh)
                node = smgr->addOctreeSceneNode(mesh->getMesh(0), 0, -1, 1024);

        if (node)
                node->setPosition(core::vector3df(-1350,-130,-1400));

        
        ICameraSceneNode * cam =  smgr->addCameraSceneNodeFPS(0,100,0.1f);
        cam->setPosition(vector3df(-20,60,-30));

        
        device->getCursorControl()->setVisible(false);

        bulletmgr = createBulletManager(device);
        bulletmgr->getWorld()->setGravity(vector3df(0,-10,0));
        bulletmgr->addTrimesh(node,0);

        int xshift,yshift,zshift;
        IMeshSceneNode * Node;
        IMeshSceneNode * Node2;
        Node = smgr->addCubeSceneNode(5,0,-1,vector3df(-20,30,0));

        Node->setMaterialType(EMT_TRANSPARENT_ADD_COLOR);
        Node->setMaterialFlag(EMF_LIGHTING,false);
        Node->setMaterialTexture(0,driver->getTexture("sphere1.jpg"));
        CIrrBPBoxBody * box= bulletmgr->addRigidBox(Node,0);
        
        Node2 = smgr->addCubeSceneNode(5,0,-1,vector3df(20,0,-20));
        Node2->setMaterialType(EMT_TRANSPARENT_ADD_COLOR);
        Node2->setMaterialFlag(EMF_LIGHTING,false);
        Node2->setMaterialTexture(0,driver->getTexture("sphere1.jpg"));
        CIrrBPBoxBody * box2 = bulletmgr->addRigidBox(Node2,40);
        
        bulletmgr->buildSlideConstraint(box,box2);

        int lastFPS = -1;

        while(device->run())
        {
                if (device->isWindowActive())
                {
                        driver->beginScene(true, true, video::SColor(255,200,200,
      200));
                        bulletmgr->stepSimulation();
                        driver->setTransform(ETS_WORLD,matrix4());

                        smgr->drawAll();
                
                        int fps = driver->getFPS();

                        if (lastFPS != fps)
                        {
                                
                                core::stringw str = L"irrBP Example - HelloWorld 
      [";
                                str += driver->getName();
                                str += "] FPS:";
                                str += fps;

                                device->setWindowCaption(str.c_str());
                                lastFPS = fps;
                        }
                        driver->endScene();
                }
                else
                        device->yield();
        }

        delete recv;
        bulletmgr->drop();
        device->drop();
        return 0;
}
\end{DoxyCode}
\hypertarget{index_linker}{}\section{Linker Settings}\label{index_linker}
Before you can compile the above example, you need to include a few libraries into your project: 
\begin{DoxyEnumerate}
\item Download BulletPhysics from \href{http://code.google.com/p/bullet/downloads/list}{\tt here} 
\item Create a solution for your compiler with cmake 
\item Compile all static libraries 
\item Import 
\begin{DoxyItemize}
\item BulletCollision.lib 
\item BulletDynamics.lib 
\item BulletSoftBody.lib 
\item LinearMath.lib 
\end{DoxyItemize}
\item Add IrrBP Source and Include directory into your project 
\end{DoxyEnumerate}